\documentclass[12pt]{extarticle}
\usepackage[utf8]{inputenc}
% \usepackage{cite}
\usepackage[a4paper, left=2cm, right=2cm, top=2cm, bottom=2cm]{geometry}
\usepackage{array}
\usepackage{comment}
\usepackage[backend=biber,style=numeric]{biblatex}
\addbibresource{references.bib}

\nocite{*}  % include all entries
\pagenumbering{arabic}

\title{Smartphone Lifecycles and E-Waste Awareness among Students}
\author{ \\ Mentor: Aneta Kartali \\ Subject: Sustainable Computing \\ Faculty of Computer and Information Science \\ University of Ljubljana}
\date{October 2025}

\begin{document}

\maketitle

\noindent This project investigates how university students manage smartphones throughout their lifecycle and how these decisions relate to environmental awareness and e-waste production. The aim is to identify student patterns in device replacement, reuse, repair, recycling and disposal, while also exploring the motivations, barriers and knowledge that shape sustainable or unsustainable behaviour. Results of the study will contribute to a better understanding of e-waste generation among young consumers and support the development of initiatives that promote responsible smartphone use, repair and recycling.

\section{Related Work}
Existing research highlights rising e-waste levels and the need for effective smartphone return and recycling systems. Studies show that convenience, awareness, and environmental attitudes significantly influence user participation, especially among young people. Work on small device take-back programs indicates that many consumers store unused phones instead of recycling them, mainly due to low knowledge and limited access to collection points. Research further shows that educational campaigns, incentives, and well-designed return schemes can increase recycling rates. This project builds on these findings by examining student behavior and identifying targeted strategies to improve sustainable smartphone lifecycles.

\section{Methodology}
The project uses a quantitative, survey-based approach to collect both behavioural and attitudinal data. The target population are undergraduate and postgraduate students across different faculties, with emphasis on FRI and FMF participants. An online questionnaire (Google Forms) will be distributed via student networks and university channels with an expected sample size of 200–300 respondents.

\noindent Data analysis will include descriptive statistics (frequencies, means, percentages), exploration of correlations between awareness levels, disposal methods and replacement motives, and construction of composite measures such as an \textit{E-Waste Behaviour Score} and an \textit{Awareness Index}. Qualitative insights from open-ended responses will be thematically analysed to assess how well students understand e-waste and sustainability issues.

\section{Milestones}
\begin{itemize}
    \item \textbf{17.10.2025} -- Submission of project proposal
    \item \textbf{05.11.2025} -- Literature review and initial survey development
    \item \textbf{26.11.2025} -- Execution of preliminary survey and behavioural assessment
    \item \textbf{19.12.2025} -- Main survey questionnaire and compilation of e-waste resources in Slovenia
    \item \textbf{12.01.2026} -- Final project report and recommendations
\end{itemize}

\begin{table}[h!]
\centering
\begin{tabular}{| m{3cm} | m{11cm}| m{1.5cm}|}
    \hline
    \textbf{Task} & \textbf{Description} & \textbf{Person} \\
    \hline
    Preliminary Survey Design & Create initial survey assessing replacement habits, disposal practices and awareness. & Nina \\
    \hline
    Survey Distribution & Share online questionnaire among university students. & Nina \\
    \hline
    Preliminary Survey Analysis & Analyse attitudes, recycling behaviour and replacement motivation trends. & Nina\\
    \hline
    Development of Main Survey & Expand question set based on early findings; include behavioural scaling items. & Nina \\
    \hline
    E-Waste Resource Mapping (Slovenia) & Compile repair shops, exchange programs and recycling centres for smartphones. & Mark\\
    \hline
    Awareness Materials & Prepare recommendations, guidelines and suggestions for student campaigns. & Mark \\
    \hline
    Final Report Writing & Summarise results, behavioural correlations and recommendations. & Nina and Mark \\
    \hline
\end{tabular}
\caption{Main project tasks with descriptions.}
\label{main_tasks}
\end{table}

\section{Expected Outcomes and Evaluation Plan}
The study will deliver a quantitative overview of sustainable vs.\ unsustainable smartphone disposal behaviour, including barriers to recycling and reuse. Awareness levels will be evaluated through descriptive metrics and composite indicators such as an E-Waste Behaviour Score and Awareness Index. The final report will include recommendations for student-focused initiatives such as repair workshops, recycling points and circular economy awareness activities. Project success will be evaluated based on clarity of survey data, sample size achieved and usefulness of proposed sustainability measures.

\printbibliography

\end{document}
