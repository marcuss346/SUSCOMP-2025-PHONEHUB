\documentclass[a4paper,12pt]{article}
\usepackage{biblatex} %Imports biblatex package
\addbibresource{references.bib} %Import the bibliography file
\usepackage[a4paper,margin=2cm]{geometry}  % sets 2 cm margins
\usepackage{graphicx}                      % for including images
\usepackage{hyperref}                      % optional: clickable ToC

\usepackage{fancyhdr} % for custom headers/footers

% Page header
\pagestyle{fancy}
\fancyhf{} % clear default header/footer
\fancyhead[L]{\textit{Smartphone Lifecycles and E-Waste Awareness Among Students}}
\fancyhead[R]{\thepage} % page number on top right
\renewcommand{\headrulewidth}{0.4pt} % line below header

\begin{document}

\begin{titlepage}
    \centering
    
    % --- Top section: Faculty name + logo ---
    \includegraphics[width=0.5\textwidth]{UL_FRI-logoENG-VER-RGB_color.png}
    
    \vspace{3cm}
    
    % --- Middle section: Title + authors ---
    {\Huge \bfseries Smartphone Lifecycles and E-Waste Awareness Among Students \par}
    \vspace{1.5cm}
    {\Large Nina Bobni\v{c} \\ Mark Ilovar \par}
    
    \vfill
    
    % --- Bottom section: Subject, professors, date ---
    {\large Subject: Sustainable Computing \par}
    {\large Professors: Aneta Kartali \par}
    \vspace{0.5cm}
    {\large January 2026 \par}
    
\end{titlepage}

% --- Table of Contents on its own page ---
\clearpage
\tableofcontents
\clearpage

% --- Sections and subsections ---
\section{Introduction}
This project investigates how university students manage smartphones throughout their lifecycle and how these decisions relate to environmental awareness and e-waste production. The aim is to identify student patterns in device replacement, reuse, repair, recycling and disposal, while also exploring the motivations, barriers and knowledge that shape sustainable or unsustainable behaviour. Results of the study will contribute to a better understanding of e-waste generation among young consumers and support the development of initiatives that promote responsible smartphone use, repair and recycling.

\section{Survey}

For this project, a survey titled \textit{Survey on the life cycle of smartphones and awareness of e-waste among students} was conducted. It was distributed among students of the Faculty of Computer and Information Science and the Faculty of Mathematics and Physics at the University of Ljubljana via student-run Discord servers that all students at the faculties have access to.

\subsection{Survey Structure}

The survey included 13 mandatory questions (14 for some participants) and 2 optional open-ended questions intended to gather students’ opinions on two topics. Most of the mandatory questions (11 out of 14) were single-choice multiple-choice questions. The remaining three were either short-answer questions or multiple-choice questions with several possible answers. \\ 
\\
The survey was divided into three sections. The first focused on demographics, asking about age, gender, faculty of study, and level of study. The second section focused on students’ mobile phone purchasing and disposal practices. The final section focused on awareness of e-waste.


\subsubsection{Data Collection}

The survey was opened for admitting answers from 20. December 2025 and closed on 5. January 2026. In total 114 students from the 2 faculties submitted answers to the survey. From these 68 students were from the Faculty of Mathematics and Physics, 46 were from the Faculty of Computer and Information Science. 

\subsubsection{Questions}


\subsection{Results}

\subsubsection{Demographics}

Of 114 participants, 29 reported being female, 78 male, and 7 either did not want to answer or answered with Other. 95 students reported being undergraduate students, 16 graduate students, and 3 doctoral students. The graph below shows the age distribution of the participants



\subsubsection{Analysis}


\subsubsection{Discussion}

\section{Alternative Solutions}
\subsection{Why people don't use buyback programs?}
From the survey resoults, we found that a lot of people don't use buyback programs.
We wanted to see what are the reasons for that. I checked some of the buyback programs and
got some interesting results. In Slovenia, there are only two buyback programs for smartphones that I found.
The first buyback program is used by Telekom Slovenije, A1, Telemach and BigBang and is called "Risajkl".
The runner od the program is a company called "Janus Trade d.o.o." and is mostly focused on promotions ran by
Samsung, since they are importing Samsung devices.

By discovering only 2 buyback programs exsist in Slovenia, we can conclude that demand for buyback programs is low.
Same could also be deducted from the survey results, where most of the students said they keep their old phones In
a drawer at home. These phones could be already recycled and their materials reused. These phones could also be dangerous
at home, since phones are usually not stored properly and batteries could leak or even explode in some cases.

By conducting further research into buyback programs in Slovenia, I found they are not used often, mainly beacuse
of poor buyback prices. For example a 3 year old Google Pixel 6a phone in almost perdect condition is bought back for only 40 euros.
Or my personal iPhone 16 pro Max, which is only 1 year old, is bought back for only 500 euros. This is a very low price, since the phone itself costs 1300 euros when bought new.



\subsection{Recycling}

Recycling is one of the crucial steps in managing e-waste from mobile phones. Old phones contain a lot of rare
earth materials that can be recycled and reused in the production of new devices. Recycling helps to reduce the need for mining new materials,
which can positively impact the enviorment by reducing the need for mining new materials, which can have significant environmental impacts. 
We can also reuse other materials like glass, plastic and metal from phones that would otherwise seat in peoples drawers or end up in landfills.

\printbibliography

\end{document}
