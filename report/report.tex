\documentclass[a4paper,12pt]{article}
\usepackage[a4paper,margin=2cm]{geometry}  % sets 2 cm margins
\usepackage{graphicx}                      % for including images
\usepackage{hyperref}                      % optional: clickable ToC

\usepackage{fancyhdr} % for custom headers/footers
\setlength{\headheight}{14.49998pt}

% Page header
\pagestyle{fancy}
\fancyhf{} % clear default header/footer
\fancyhead[L]{\textit{Smartphone Lifecycles and E-Waste Awareness Among Students}}
\fancyhead[R]{\thepage} % page number on top right
\renewcommand{\headrulewidth}{0.4pt} % line below header

\begin{document}

\begin{titlepage}
    \centering
    
    % --- Top section: Faculty name + logo ---
    \includegraphics[width=0.5\textwidth]{fri-logo.png}
    
    \vspace{3cm}
    
    % --- Middle section: Title + authors ---
    {\Huge \bfseries Smartphone Lifecycles and E-Waste Awareness Among Students \par}
    \vspace{1.5cm}
    {\Large Nina Bobni\v{c} \\ Mark Ilovar \par}
    
    \vfill
    
    % --- Bottom section: Subject, professors, date ---
    {\large Subject: Sustainable Computing \par}
    {\large Professors: Aneta Kartali \par}
    \vspace{0.5cm}
    {\large January 2026 \par}
    
\end{titlepage}

% --- Table of Contents on its own page ---
\clearpage
\tableofcontents
\clearpage

% --- Sections and subsections ---
\section{Introduction}
This project investigates how university students manage smartphones throughout their lifecycle and how these decisions relate to environmental awareness and e-waste production. The aim is to identify student patterns in device replacement, reuse, repair, recycling and disposal, while also exploring the motivations, barriers and knowledge that shape sustainable or unsustainable behaviour. Results of the study will contribute to a better understanding of e-waste generation among young consumers and support the development of initiatives that promote responsible smartphone use, repair and recycling.

\section{Survey}

\subsection{Survey Structure}

\subsubsection{Methodology}


\subsubsection{Questions}
The questions in the survey distributed to students with answers were as follows:
 \begin{enumerate}
     \item Question1
     \begin{itemize}
         \item answer1
         \item answer2
     \end{itemize}
     \item Question2
 \end{enumerate}

\subsection{Results}

\subsubsection{Analysis}


\subsubsection{Discussion}

\section{Alternative Solutions}
\subsection{Why people don't use buyback programs?}
From the survey resoults, we found that a lot of people don't use buyback programs.
We wanted to see what are the reasons for that. I checked some of the buyback programs and
got some interesting results. In Slovenia, there are only two buyback programs for smartphones that I found.
The first buyback program is used by Telekom Slovenije, A1, Telemach and BigBang and is called "Risajkl".
The runner od the program is a company called "Janus Trade d.o.o." and is mostly focused on promotions ran by
Samsung, since they are importing Samsung devices.

\subsection{Recycling}

Recycling is one of the crucial steps in managing e-waste from mobile phones. Old phones contain a lot of rare
earth materials that can be recycled and reused in the production of new devices. Recycling helps to reduce the need for mining new materials,
which can positively impact the enviorment by reducing the need for mining new materials, which can have significant environmental impacts. 
We can also reuse other materials like glass, plastic and metal from phones that



\end{document}
