\documentclass[12pt]{extarticle}
\usepackage[utf8]{inputenc}
\usepackage{cite}
\usepackage[a4paper, left=2cm, right=2cm, top=2cm, bottom=2cm]{geometry}
\usepackage{array}
\usepackage{comment}
\pagenumbering{arabic} % Use Arabic numerals for page numbers

\title{Impact of a Phone's Price Range on Its Lifespan and Carbon Footprint}
\author{Mark Ilovar, Nina Bobnič \\
Mentor: Aneta Kartali \\
Subject: Sustainable Computing \\
Faculty of Computer and Information Science \\ University of Ljubljana \\}
\date{October 2025}

\begin{document}

\maketitle

\noindent This project will mostly focus on the impact of a phone's proce range on its lifespan and carbon footprint. The project will try to provide useful information on how to become a more environmentally conscious consumer when buying a new phone. A big part of all the environmental movement towards a better future are EU and the UN that are pushing for more sustainable practices in product development and production.

\section{Related Work}
Some good data on carbon footprint of phones can be found in books such as \textit{The carbon footprint of everything} \cite{bernerslee2010} and \textit{Sustainable Energy - without the hot air} \cite{mackay2009}. These books include calculated carbon footprints of various devices, including phones, and provide a good starting point for understanding the environmental impact of these devices. We believe some useful data and information can be found on company websites such as Apple  \cite{apple_environment} and Samsung Electronics \cite{samsung_climate_action} as these global companys are trying to follow the sustainability trend. None of these efforts would be made without the EU and European Commission pushing for more Sustainable practices in product development and production. Some important research by the European Commission includes research on the impacts of the \textit{Right to Repair} policy \cite{europeancommission_righttorepair} and enforcing the new \textit{Energy Label} \cite{eu_energylabel}.

\section{Methodology}
The methodology of this project will include determining phone price ranges and selecting the most representative phone model for each price range. For this part we will use data from company websites such as Apple, Samsung and similar companies. Data will then be gathered on the lifespan of each device (including years of software support, battery health and repairability). \\

\noindent Additionally, data on carbon footprint of each device will be collected and compared between price ranges. For this part we will use data from books such as \textit{The carbon footprint of everything} \cite{bernerslee2010} and \textit{Sustainable Energy - without the hot air} \cite{mackay2009}, as well as data from the ILCD International Life Cycle Data system \cite{ilcd_system} and IPCC reports \cite{ipcc}. \\

\noindent The project will also include lifecycle mapping of the selected phone models and some estimations of carbon footprint at each stage. Useful data for this part of the project can be found in the ILCD INternational Lufe Cycle Data system \cite{ilcd_system}. Additionally, the explanation of the \textit{Energy Labels} \cite{eu_energylabel} will be provided and some comparisons of company promises versus real world data will be made. These promises can be found on environmental parts of company websites such as Apple \cite{apple_environment} and Samsung \cite{samsung_climate_action}. \\

\section{Milestones}
The proposed timeline for the project is as follows:
\begin{itemize}
    \item \textbf{17.10.2025} -- Submission of project proposal
    
    \item \textbf{05.11.2025} -- Completion of literature review and lifecycle framework
    
    \item \textbf{26.11.2025} -- Data collection, preliminary comparisons and carbon footprint estimations
    
    \item \textbf{19.12.2025} -- Draft of price range comparison and carbon footprint analysis
    
    \item \textbf{12.01.2026} -- Submission of final project report
\end{itemize}

Some main tasks in developing this project are shown in the table bellow:

\begin{table}[h!]
\centering
\begin{tabular}{| m{3cm} | m{11cm}| m{1.5cm}|}
    \hline
    \textbf{Task} & \textbf{Content} & \textbf{Person}\\
    \hline
    Brand selection & Select a smartphone brand to analyze. & Mark\\
    \hline
    Price range definition & Define price ranges for analysis (e.g., budget, mid-range, flagship). & Mark\\
    \hline
    Model selection & Choose representative models from each price range. & Mark\\
    \hline
    Data collection & Gather data on lifespan, software support and repairability. & Nina\\
    \hline
    Lifecycle mappinig & Map out the lifecycle stages of each selected model (extraction, manufacturing, shipping, use, disposal). & Mark\\
    \hline
    Carbon footprint estimation & Estimate the carbon footprint for each lifecycle stage of the selected models. & Nina\\
    \hline
    Data analysis & Analyze the collected data to identify trends and correlations between price range, lifespan, and carbon footprint. & Nina\\
    \hline
    Recomendations & Provide recommendations for consumers based on the findings. & Mark, Nina\\
    \hline
    Report writing & Compile the findings into a comprehensive report. & Mark, Nina\\
    \hline
\end{tabular}
\caption{Table of the main tasks of the project with their content.}
\label{main_tasks}
\end{table}

\section{Evaluation Plan}
The evaluation will consider how the procejt aligns with the UN Sustainable Development Goals (SDGs) \cite{UN_SDGS}, particulary goals 8 (Decent work and economic growth), 12 (Responsible consumption and production) and 13 (Climate action). The project'ssuccess will be evaluated based on the reliability and availability of data sources and the quality of the carbon footprint analysis. A sadisfactory project should include explicid recomendations fot the end user on how to choose a phone based on its price range, lifespan and environmental impact based on carbon footprint of the device.

\begin{comment}
\section{Resoureces}
Key resources for the project include:
\begin{itemize}
    \item company websites (Apple \cite{apple_environment}, Samsung \cite{samsung_climate_action})
    \item books (The carbon footprint of everything \cite{bernerslee2010}, Sustainable Energy - without the hot air \cite{mackay2009})
    \item European Commission's "Right to Repair" policy reports \cite{europeancommission_righttorepair}
    \item EU's data on the Energy label \cite{eu_energylabel}
    \item UN Sustainable Development Goals (SDGs) documentation \cite{UN_SDGS}
    \item IPCC reports and data on carbon footprint \cite{ipcc}
    \item ILCD International Life Cycle Data system \cite{ilcd_system}
    \item Academic databases
    \item Tools for data analysis
\end{itemize}
\end{comment}

\bibliographystyle{plain}
\bibliography{references} 

\end{document}